\documentclass[12pt, a4paper, twocolumn]{article}
\usepackage[utf8]{inputenc}
\usepackage{graphicx}
\usepackage{geometry}
\usepackage{hyperref}
\usepackage{booktabs}
\usepackage{float}
\usepackage{caption}
\usepackage{subcaption}
\usepackage{amsmath}
\usepackage{amssymb}
\usepackage{cite}
\usepackage{tabularx}
\usepackage{placeins}

\geometry{
 a4paper,
 total={170mm,257mm},
 left=20mm,
 top=20mm,
}
\setlength{\columnsep}{10mm}

\title{\textbf{Clustering and Classifying Livestock Farming Profiles to Identify Agricultural Vulnerability in Nepal: Using K-Means Clustering and Decision Tree Classification}}
\author{
    \textbf{Eklak Dangaura} \\
    MSc. Data Science and Computational Intelligence \\
    Softwarica College of IT \& E-Commerce (Coventry University) \\
    Student ID: 250145 - Coventry ID: 16542457
}
\date{}

\begin{document}

\maketitle

\begin{abstract}
This report employs a machine learning approach to analyze livestock farming profiles and identify agricultural vulnerability in Nepal. Utilizing a synthetic dataset spanning 50 years (1972-2021), the study applies K-Means clustering to identify distinct farming typologies and track their evolution. A Decision Tree classifier is developed to categorize these profiles based on key agricultural indicators. By examining structural shifts—such as land fragmentation and the transition to commercial poultry—the analysis highlights changing vulnerability patterns across districts. Furthermore, forecasting models project future trends to 2030, providing insights for targeted policy interventions.
\end{abstract}

\section{Introduction}
Livestock farming is a cornerstone of Nepal's agricultural economy, contributing significantly to the national GDP and providing livelihoods for a vast majority of the rural population. The sector's diversity mirrors Nepal's varied topography, ranging from the Terai plains suitable for intensive farming to the high Himalayas where livestock is integral to survival. However, the sector faces multifaceted challenges, including climate change, urbanization, and resource constraints. Understanding the long-term trends and structural shifts in this sector is essential for effective policy formulation and sustainable development.

Understanding the structural dynamics of livestock farming is crucial for assessing agricultural vulnerability in Nepal. This study leverages machine learning techniques to reconstruct and analyze the agricultural landscape from 1972 to 2021. By applying clustering and classification algorithms to a synthetically generated dataset, we aim to:
\begin{itemize}
    \item Analyze temporal trends in livestock populations, land holding sizes, and irrigation coverage.
    \item Identify distinct farming profiles using unsupervised learning (clustering) and track their evolution.
    \item Quantify structural changes in farming systems at the district level.
    \item Forecast future trends for the decade 2022-2030.
    \item Develop a predictive model to classify farming systems based on agricultural features.
\end{itemize}

\section{Literature Review}
The evolution of livestock farming in developing economies has been a subject of extensive research. Studies have highlighted the transition from subsistence to commercial farming as a critical pathway for rural development. In the context of Nepal, previous research has focused on specific livestock sectors, such as dairy or poultry, but comprehensive long-term temporal analyses are limited due to data scarcity.

Quantitative approaches to characterizing Nepal's farming systems are gaining traction. Zepeda et al. (2023) used Principal Component Analysis (PCA) and Hierarchical Clustering to classify Western Nepalese households \cite{zepeda2023}, finding that structural factors—specifically livestock and land ownership—are robust indicators of farm typology. This work reflects a broader global shift toward using machine learning for agricultural pattern recognition.

While Zepeda et al. focused on cross-sectional typology, this study contributes to the literature by extending such quantitative classification to a longitudinal dimension. By applying K-Means clustering to a synthetically reconstructed 50-year dataset, we offer a novel perspective on the temporal evolution and structural transformation of Nepalese agriculture.

\section{Methodology}

\subsection{Dataset Description}
The analysis is grounded in three primary datasets representing the 2021 baseline, sourced from agricultural censuses.

\begin{table}[H]
\centering
\caption{Dataset 1: Holdings and Area}
\small
\begin{tabularx}{\columnwidth}{@{}Xl@{}}
\toprule
\textbf{Column Name} & \textbf{Data Type} \\ \midrule
\texttt{Districts} & Text \\
\texttt{Number of holdings} & Integer \\
\texttt{Total wet area (ha)} & Decimal \\
\texttt{Total dry area (ha)} & Decimal \\
\texttt{Total area (ha)} & Decimal \\ \bottomrule
\end{tabularx}
\end{table}

\begin{table}[H]
\centering
\caption{Dataset 2: Irrigation Data}
\small
\begin{tabularx}{\columnwidth}{@{}Xl@{}}
\toprule
\textbf{Column Name} & \textbf{Data Type} \\ \midrule
\texttt{Districts} & Text \\
\texttt{Total no. of holdings} & Integer \\
\texttt{Total area (ha)} & Decimal \\
\texttt{No. of holdings reporting irrigation} & Integer \\
\texttt{Total area (ha) of irrigation} & Decimal \\
\texttt{Irrigation by Gravity} & Integer/Decimal \\
\texttt{Irrigation by Pumping} & Integer/Decimal \\
\texttt{Irrigation by Dam/Reservoir} & Integer/Decimal \\
\texttt{Irrigation by Tubewell/Boring} & Integer/Decimal \\
\texttt{Irrigation by Other Types} & Integer/Decimal \\
\texttt{Irrigation by Mixed Types} & Integer/Decimal \\ \bottomrule
\end{tabularx}
\end{table}

\begin{table}[H]
\centering
\caption{Dataset 3: Livestock Data}
\small
\begin{tabularx}{\columnwidth}{@{}Xl@{}}
\toprule
\textbf{Column Name} & \textbf{Data Type} \\ \midrule
\texttt{District} & Text \\
\texttt{Total number of holdings} & Integer \\
\texttt{Holdings reporting livestock} & Integer \\
\texttt{Cattle (Holdings/Count)} & Integer \\
\texttt{Buffalo (Holdings/Count)} & Integer \\
\texttt{Goat/Chyangra (Holdings/Count)} & Integer \\
\texttt{Sheep (Holdings/Count)} & Integer/Decimal \\
\texttt{Pig/Boar (Holdings/Count)} & Integer \\
\texttt{Other Livestocks (Holdings/Count)} & Integer \\
\texttt{Poultry (Holdings/Count)} & Integer \\
\texttt{Other Birds (Holdings/Count)} & Integer \\ \bottomrule
\end{tabularx}
\end{table}

These datasets were merged and processed to generate the following derived features for analysis:

\begin{table}[H]
\centering
\caption{Derived Features Used in Analysis}
\label{tab:features}
\small
\begin{tabularx}{\columnwidth}{@{}Xl@{}}
\toprule
\textbf{Feature Name} & \textbf{Description} \\ \midrule
\texttt{Avg\_Land\_Size} & Average land holding size (ha) \\
\texttt{Pct\_Irrigated} & Percentage of irrigated land \\
\texttt{Cattle\_per\_HH} & Average cattle per household \\
\texttt{Buffalo\_per\_HH} & Average buffalo per household \\
\texttt{Goat\_per\_HH} & Average goats per household \\
\texttt{Pig\_per\_HH} & Average pigs per household \\
\texttt{Poultry\_per\_HH} & Average poultry per household \\ \bottomrule
\end{tabularx}
\end{table}

\subsection{Data Preprocessing}
Data preprocessing was a critical step to ensure the integrity of the analysis, involving several distinct stages to handle the heterogeneity of the source datasets.

\subsubsection{Standardization and Merging}
District names across the three source datasets exhibited inconsistencies in casing and whitespace (e.g., "Kathmandu" vs " Kathmandu "). A standardization function was applied to strip leading/trailing whitespace and convert all names to lowercase, ensuring accurate matching keys. The datasets were then merged using an outer join strategy to preserve all available district information. A mismatch analysis was conducted to identify districts present in some datasets but not others, ensuring that the final baseline covered the maximum possible geographic area.

\subsubsection{Missing Data and Outlier Handling}
Missing values were handled based on the nature of the data:
\begin{itemize}
    \item \textbf{Imputation}: Non-numeric entries in livestock counts were coerced to NaN and subsequently imputed with zero, under the assumption that a missing report indicates zero presence of that specific livestock type.
    \item \textbf{Infinite Values}: During feature engineering (e.g., calculating livestock per household), division by zero scenarios (where household count was zero or missing) resulted in infinite values. These were identified and replaced with zero to maintain numerical stability.
    \item \textbf{Cleanup}: Rows lacking a valid district identifier after merging were removed from the baseline dataset.
\end{itemize}

\subsubsection{Feature Scaling}
To prepare the data for K-Means clustering, which is distance-based and sensitive to the scale of input variables, feature standardization was applied. The \texttt{StandardScaler} from the Scikit-learn library was used to transform features such as \textit{Avg\_Land\_Size} (measured in hectares) and \textit{Poultry\_per\_HH} (count ratio) into a common scale with a mean of 0 and a standard deviation of 1. This ensured that features with larger magnitudes did not disproportionately influence the cluster formation. The merged dataset, after these preprocessing steps, served as the robust seed for the synthetic data generation process.

\begin{figure}[H]
    \centering
    \includegraphics[width=\columnwidth]{Results/01-Correlation-Comparison.png}
    \caption{Feature Correlation Matrix (2021 Baseline)}
    \label{fig:correlation}
\end{figure}

\subsection{Synthetic Data Generation (Backcasting)}
To enable a longitudinal analysis, synthetic historical data was generated for the period 1972-2020 using a backcasting approach starting from the 2021 baseline. The generation process applied specific annual trend assumptions to simulate historical conditions:
\begin{itemize}
    \item \textbf{Poultry}: Assumed to have grown significantly over time. Backcasting applied a factor of $0.92$ per year (approx. -8\% annual change going backwards).
    \item \textbf{Land Size}: Assumed to have fragmented over time. Backcasting applied a factor of $1.008$ per year (+0.8\% annual change going backwards), implying larger landholdings in the past.
    \item \textbf{Irrigation}: Assumed to have expanded. Backcasting applied a factor of $0.99$ per year (-1\% annual change going backwards).
    \item \textbf{Livestock (Pigs)}: Applied a factor of $0.98$ per year.
    \item \textbf{Livestock (Cattle, Buffalo, Goats)}: Applied a factor of $1.001$ per year, assuming relatively stable or slightly declining populations relative to households in the past.
\end{itemize}
Random uniform noise ($\pm 2\%$) was added at each step to introduce realistic variability.

\subsection{Clustering Analysis}
Unsupervised learning was employed to categorize districts into distinct farming profiles.
\begin{itemize}
    \item \textbf{Feature Selection}: Key features included \textit{Avg\_Land\_Size}, \textit{Pct\_Irrigated}, \textit{Cattle\_per\_HH}, \textit{Buffalo\_per\_HH}, \textit{Goat\_per\_HH}, and \textit{Poultry\_per\_HH}.
    \item \textbf{Preprocessing}: Data was standardized using \texttt{StandardScaler} to ensure all features contributed equally.
    \item \textbf{Algorithm}: K-Means clustering was applied to the full 50-year dataset.
    \item \textbf{Optimal K}: The optimal number of clusters was determined using the Elbow Method and Silhouette Score analysis. The analysis yielded the highest silhouette score (0.317) for $k=9$, which was selected as the optimal number of clusters.

    \begin{figure}[H]
        \centering
        \includegraphics[width=\columnwidth]{Results/08-Best-K-Find.png}
        \caption{Elbow Method and Silhouette Analysis}
        \label{fig:elbow}
    \end{figure}

    \item \textbf{Cluster Naming}: Clusters were dynamically named based on dominant features. This resulted in 9 distinct profiles ranging from "Mixed / Diversified" to specialized categories like "Commercial Poultry \& Highland/Goat".
\end{itemize}

\begin{figure*}[htbp]
    \centering
    \includegraphics[width=0.95\textwidth]{Results/10-Cluster-HeatMaps.png}
    \caption{Cluster Profiles Heatmap (Z-Scores)}
    \label{fig:cluster_heatmap}
\end{figure*}

\subsection{Structural Change Detection}
To quantify the magnitude of change in farming systems, a Structural Change Index was calculated for each district. This index compares the mean values of agricultural indicators between the first decade (1972-1982) and the last decade (2012-2021).
\begin{equation}
    \text{Change Index} = \frac{1}{N} \sum_{i=1}^{N} \left| \frac{\mu_{late, i} - \mu_{early, i}}{\mu_{early, i}} \right| \times 100
\end{equation}
where $\mu_{late, i}$ and $\mu_{early, i}$ are the mean values of feature $i$ in the late and early periods, respectively.

\subsection{Predictive Modeling}
\subsubsection{Forecasting (2022-2030)}
Linear Regression models were trained on the national annual means of each feature from 1972 to 2021. These models were then used to forecast trends for the years 2022 through 2030, providing a glimpse into the potential future state of Nepal's agriculture.

\subsubsection{Classification}
A Decision Tree classifier was developed to predict the farming cluster of a district based on its agricultural features.
\begin{itemize}
    \item \textbf{Dataset}: The full 50-year dataset with cluster labels assigned by K-Means.
    \item \textbf{Split}: 80\% training and 20\% testing, stratified by cluster.
    \item \textbf{Model Parameters}: To ensure model generalizability and prevent overfitting, the Decision Tree was configured with a maximum depth of 5 (\texttt{max\_depth=5}), a minimum of 10 samples required to split an internal node (\texttt{min\_samples\_split=10}), and a minimum of 5 samples per leaf node (\texttt{min\_samples\_leaf=5}).
    \item \textbf{Evaluation}: The model was evaluated using accuracy, confusion matrix, and a detailed classification report including precision, recall, and F1-score for each cluster.
\end{itemize}

\section{Results}

\subsection{Temporal Trends (1972-2021)}
The analysis of national trends reveals distinct patterns in Nepal's agricultural evolution.
\begin{itemize}
    \item \textbf{Poultry Revolution}: A sharp exponential increase in poultry per household is observed, reflecting the commercialization of this sector.
    \item \textbf{Land Fragmentation}: Average land holding sizes have consistently decreased, consistent with population growth and land division.
    \item \textbf{Irrigation Expansion}: The percentage of irrigated land has shown a steady upward trend.
\end{itemize}

\textbf{50-Year Change Analysis (1972 $\to$ 2021):}
\begin{itemize}
    \item Avg\_Land\_Size: $0.75 \to 0.51$ (-31.6\%)
    \item Pct\_Irrigated: $25.59 \to 42.14$ (+64.6\%)
    \item Poultry\_per\_HH: $0.18 \to 10.57$ (+5526.7\%)
    \item Pig\_per\_HH: $0.12 \to 0.32$ (+155.4\%)
\end{itemize}

\begin{table}[H]
\centering
\caption{Standard Deviation by Decade}
\resizebox{\columnwidth}{!}{%
\begin{tabular}{@{}lccccccc@{}}
\toprule
Decade & Land & Irrig & Cattle & Buff & Goat & Pig & Pltry \\ \midrule
1970 & 0.23 & 15.38 & 1.05 & 0.36 & 1.69 & 0.17 & 0.23 \\
1980 & 0.21 & 16.84 & 1.05 & 0.35 & 1.69 & 0.21 & 0.49 \\
1990 & 0.19 & 18.64 & 1.04 & 0.35 & 1.66 & 0.25 & 1.12 \\
2000 & 0.18 & 20.83 & 1.01 & 0.35 & 1.65 & 0.31 & 2.57 \\
2010 & 0.16 & 23.10 & 0.99 & 0.34 & 1.59 & 0.37 & 5.96 \\
2020 & 0.16 & 24.44 & 0.99 & 0.34 & 1.56 & 0.42 & 8.98 \\ \bottomrule
\end{tabular}%
}
\end{table}

\begin{table}[H]
\centering
\caption{Decade-over-Decade Growth Rates (\%)}
\resizebox{\columnwidth}{!}{%
\begin{tabular}{@{}lrrrrrrr@{}}
\toprule
Decade & Land & Irrig & Cattle & Buff & Goat & Pig & Pltry \\ \midrule
1980 & -6.8 & 9.5 & -0.5 & -0.8 & -0.7 & 20.5 & 113.3 \\
1990 & -7.6 & 10.7 & -1.2 & -0.5 & -1.0 & 22.4 & 130.1 \\
2000 & -7.6 & 10.8 & -1.3 & -0.8 & -0.9 & 22.7 & 130.5 \\
2010 & -7.7 & 10.9 & -1.3 & -1.0 & -1.3 & 21.8 & 130.7 \\
2020 & -4.6 & 6.2 & -0.6 & -0.6 & -0.7 & 12.5 & 60.1 \\ \bottomrule
\end{tabular}%
}
\end{table}

\begin{figure*}[htbp]
    \centering
    \includegraphics[width=0.95\textwidth]{Results/14-Visualize-Forecasting.png}
    \caption{Agricultural Trends: Historical (1972-2021) and Forecast (2022-2030)}
    \label{fig:trends}
\end{figure*}

\subsection{Cluster Evolution and Transition}
The K-Means analysis identified distinct farming clusters, revealing a dynamic shift in Nepal's agricultural landscape. The evolution of these clusters over time shows a clear transition from traditional, subsistence-based systems towards more specialized and commercialized forms of agriculture.

The identified clusters include:
\begin{itemize}
    \item \textbf{Cluster 0 (Mixed / Diversified)}: Good irrigation, diversified livestock.
    \item \textbf{Cluster 1 (Buffalo/Dairy)}: High buffalo density, traditional dairy focus.
    \item \textbf{Cluster 2 (Irrigated \& Large Land)}: Large landholdings with high irrigation coverage.
    \item \textbf{Cluster 3 (Large Land \& Highland/Goat+)}: Large land sizes with significant goat populations.
    \item \textbf{Cluster 4 (Highland/Goat)}: Dominated by goat farming, typical of hill districts.
    \item \textbf{Cluster 5 (Commercial Poultry \& Irrigated)}: Emerging commercial hubs with high poultry density.
    \item \textbf{Cluster 6 (Cattle)}: High cattle density, traditional systems.
    \item \textbf{Cluster 7 (Highland/Goat \& Cattle+)}: Intensive grazing systems.
    \item \textbf{Cluster 8 (Commercial Poultry \& Highland/Goat)}: Mixed commercial poultry with traditional goat farming.
\end{itemize}

\begin{table}[H]
\centering
\caption{Era Profiles (National Averages)}
\resizebox{\columnwidth}{!}{%
\begin{tabular}{@{}lcccc@{}}
\toprule
Era & Land & Irrig & Goat & Poultry \\ \midrule
Traditional (72-85) & 0.71 & 27.31 & 3.86 & 0.32 \\
Early Modern (86-00) & 0.64 & 31.65 & 3.81 & 1.09 \\
Commercial (01-21) & 0.55 & 38.10 & 3.73 & 5.21 \\ \bottomrule
\end{tabular}%
}
\end{table}

The Cluster Transition Matrix (1972 vs 2021) highlights that while some districts have maintained their traditional profiles, a significant number have shifted to new clusters, particularly those associated with commercial poultry and intensive farming. This transition is indicative of the broader economic shifts occurring in the country.

\begin{figure*}[htbp]
    \centering
    \begin{subfigure}{0.48\textwidth}
        \includegraphics[width=\linewidth]{Results/11-Cluster-Evolution.png}
        \caption{Cluster Evolution}
    \end{subfigure}
    \hfill
    \begin{subfigure}{0.48\textwidth}
        \includegraphics[width=\linewidth]{Results/12-Cluster-Transition-Matrix.png}
        \caption{Transition Matrix (1972 $\to$ 2021)}
    \end{subfigure}
    \caption{Evolution and Transition of Farming Clusters}
    \label{fig:transition}
\end{figure*}

\subsection{Structural Change Analysis}
The Structural Change Index identifies districts that have undergone the most radical transformations. These districts are often those that have rapidly urbanized or adopted commercial farming practices. The top districts exhibit high indices, driven largely by shifts in livestock composition (e.g., from large ruminants to poultry) and land use intensity.

The top 5 districts by Structural Change Index are:
\begin{enumerate}
    \item Kanchanpur (522.8)
    \item Rautahat (509.4)
    \item Parbat (502.4)
    \item Jhapa (498.2)
    \item Panchthar (494.5)
\end{enumerate}

\begin{figure*}[htbp]
    \centering
    \includegraphics[width=0.95\textwidth]{Results/13-Structure-Change-Visualization.png}
    \caption{Top 20 Districts by Structural Change (1972-2021)}
    \label{fig:structural_change}
\end{figure*}

\subsection{Geospatial Analysis}
The geospatial analysis visualizes the distribution of farming clusters and structural changes across Nepal's districts. Figure \ref{fig:thematic_map} illustrates the geographic spread of the identified farming clusters in 2021 and the intensity of structural change over the 50-year period.

\begin{figure*}[htbp]
    \centering
    \includegraphics[width=0.95\textwidth]{Results/19-Thematic-Map-Nepal.png}
    \caption{Geospatial Analysis: Farming Clusters (2021) and Structural Change Index (1972-2021)}
    \label{fig:thematic_map}
\end{figure*}

\subsection{Forecasting (2022-2030)}
Forecasts for the next decade suggest a continuation of current trends. Poultry production is projected to grow further, while land holding sizes are expected to continue their gradual decline. These projections highlight the need for policies that support intensive farming on smaller land plots.

\begin{table}[H]
\centering
\caption{National Forecast (2022-2030)}
\resizebox{\columnwidth}{!}{%
\begin{tabular}{@{}lcccc@{}}
\toprule
Year & Land & Irrig & Goat & Poultry \\ \midrule
2022 & 0.50 & 41.74 & 3.69 & 7.01 \\
2024 & 0.49 & 42.42 & 3.68 & 7.35 \\
2026 & 0.48 & 43.09 & 3.68 & 7.70 \\
2028 & 0.47 & 43.76 & 3.67 & 8.05 \\
2030 & 0.46 & 44.44 & 3.66 & 8.39 \\ \bottomrule
\end{tabular}%
}
\end{table}

\subsection{Classification Model Performance}
The Decision Tree classifier achieved a training accuracy of 90.94\% and a testing accuracy of 89.35\%. The confusion matrix indicates that the model can effectively distinguish between distinct farming profiles.

Feature importance analysis confirms that \textit{Pct\_Irrigated} (25.1\%) and \textit{Avg\_Land\_Size} (23.2\%) are the most critical determinants, followed by \textit{Cattle\_per\_HH} (21.4\%).

\begin{table}[H]
\centering
\caption{Classification Report (Test Set)}
\resizebox{\columnwidth}{!}{%
\begin{tabular}{@{}lccc@{}}
\toprule
Cluster & Precision & Recall & F1-Score \\ \midrule
C0: Mixed & 0.97 & 1.00 & 0.99 \\
C1: Buffalo/Dairy & 1.00 & 0.89 & 0.94 \\
C2: Irrigated & 0.95 & 0.99 & 0.97 \\
C3: Large Land & 0.93 & 0.80 & 0.86 \\
C4: Highland/Goat & 0.75 & 0.96 & 0.84 \\
C5: Poultry & 1.00 & 1.00 & 1.00 \\
C6: Cattle & 1.00 & 0.89 & 0.94 \\
C7: Highland+ & 1.00 & 1.00 & 1.00 \\
C8: Poultry+ & 0.77 & 0.36 & 0.49 \\ \bottomrule
\end{tabular}%
}
\end{table}

\begin{figure*}[htbp]
    \centering
    \begin{subfigure}{0.48\textwidth}
        \includegraphics[width=\linewidth]{Results/15-Confusion-Matrix-Decision-Tree.png}
        \caption{Confusion Matrix}
    \end{subfigure}
    \hfill
    \begin{subfigure}{0.48\textwidth}
        \includegraphics[width=\linewidth]{Results/17-DecisionTree-Feature-Importance.png}
        \caption{Feature Importance}
    \end{subfigure}
    \caption{Decision Tree Classifier Performance}
    \label{fig:classification_performance}
\end{figure*}

\begin{figure}[H]
    \centering
    \includegraphics[width=\columnwidth]{Results/18-Model-Performance-Analysis-By-Era.png}
    \caption{Classification Accuracy by Era}
    \label{fig:era_performance}
\end{figure}

\section{Discussion}
The study successfully reconstructs the trajectory of Nepalese agriculture. The shift from large-animal based systems to poultry and smaller livestock reflects adaptation to shrinking land sizes and changing market demands. The stability analysis reveals that while some regions are resilient (or stagnant), others are dynamic and rapidly evolving.

Our findings resonate with the work of Zepeda et al. (2023), who identified distinct farm typologies in Western Nepal driven by structural variables. Similarly, our K-Means analysis confirms that livestock composition and land size are robust discriminators for farming systems. However, our temporal analysis adds a dynamic layer, showing that these "structural" variables are not static over decades but evolve in response to broader economic forces. The high accuracy of our classification model further validates the existence of distinct, predictable farming profiles, suggesting that policy interventions can be targeted effectively based on these typologies.

\section{Policy Recommendations}
Based on the identified farming profiles, targeted interventions are recommended:
\begin{itemize}
    \item \textbf{Cluster 0 (Mixed)}: Establish livestock insurance and disease surveillance.
    \item \textbf{Cluster 1 (Buffalo/Dairy)}: Implement small-scale irrigation and micro-credit programs.
    \item \textbf{Cluster 2 (Irrigated)}: Support traditional transhumance and highland breed conservation.
    \item \textbf{Cluster 3 (Large Land)}: Promote integrated farming and small-scale dairy cooperatives.
    \item \textbf{Cluster 5 \& 8 (Commercial Poultry)}: Monitor production trends and ensure biosecurity.
\end{itemize}

\section{Conclusion and Future Work}
This study successfully applied machine learning techniques to cluster and classify livestock farming profiles, providing a framework to assess agricultural vulnerability across Nepal's districts (1972-2021).

\textbf{Key Achievements:}
\begin{itemize}
    \item \textbf{Data Integration}: Merged 2021 baseline data and generated 50 years of synthetic history.
    \item \textbf{Clustering}: Identified 9 distinct farming profiles with a silhouette score of 0.317.
    \item \textbf{Classification}: Developed a Decision Tree model with 89.4\% testing accuracy.
    \item \textbf{Forecasting}: Projected trends to 2030, highlighting continued poultry growth.
\end{itemize}

The findings underscore the structural transformation of the sector, particularly the shift towards commercial poultry and the challenges of land fragmentation. Future work should focus on integrating real historical data and incorporating climate indicators.

\FloatBarrier

\begin{thebibliography}{9}
\bibitem{fao}
FAO. (2021). \textit{World Food and Agriculture - Statistical Yearbook 2021}. Food and Agriculture Organization of the United Nations.

\bibitem{moald}
Ministry of Agriculture and Livestock Development (MoALD). (2021). \textit{Statistical Information on Nepalese Agriculture}. Government of Nepal.

\bibitem{sklearn}
Pedregosa, F., et al. (2011). Scikit-learn: Machine Learning in Python. \textit{Journal of Machine Learning Research}, 12, 2825-2830.

\bibitem{kmeans}
Jain, A. K. (2010). Data clustering: 50 years beyond K-means. \textit{Pattern Recognition Letters}, 31(8), 651-666.

\bibitem{zepeda2023}
Zepeda, A., et al. (2023). \textit{Characterizing farm households in Surkhet, Western Nepal, through a quantitative farming systems typology}. CIMMYT.

\end{thebibliography}

\appendix
\onecolumn

\section{Appendix: Code Implementation}
The google colab notebook for the execution of this work can be found here(\url{https://colab.research.google.com/drive/1aerGKYqJ1B1fSHqObaxG8xfSssOjG7rp?usp=sharing}).

\centering
\begin{figure}[H]
\centering
\includegraphics[width=0.95\textwidth]{Code/1.1-Install-Libraries.png}
\caption{1.1-Install-Libraries}
\end{figure}
\FloatBarrier
\begin{figure}[H]
\centering
\includegraphics[width=0.95\textwidth]{Code/1.2-Import-Libraries.png}
\caption{1.2-Import-Libraries}
\end{figure}
\FloatBarrier
\begin{figure}[H]
\centering
\includegraphics[width=0.95\textwidth]{Code/2.0.1-Load-2021-BaseLine-Data.png}
\caption{2.0.1-Load-2021-BaseLine-Data}
\end{figure}
\FloatBarrier
\begin{figure}[H]
\centering
\includegraphics[width=0.95\textwidth]{Code/2.0.2-Define-File-Paths.png}
\caption{2.0.2-Define-File-Paths}
\end{figure}
\FloatBarrier
\begin{figure}[H]
\centering
\includegraphics[width=0.95\textwidth]{Code/2.0.3-Load-CSV-Files.png}
\caption{2.0.3-Load-CSV-Files}
\end{figure}
\FloatBarrier
\begin{figure}[H]
\centering
\includegraphics[width=0.95\textwidth]{Code/2.1.1-EDA-HouseHolds.png}
\caption{2.1.1-EDA-HouseHolds}
\end{figure}
\FloatBarrier
\begin{figure}[H]
\centering
\includegraphics[width=0.95\textwidth]{Code/2.1.2-EDA-Irrigation.png}
\caption{2.1.2-EDA-Irrigation}
\end{figure}
\FloatBarrier
\begin{figure}[H]
\centering
\includegraphics[width=0.95\textwidth]{Code/2.1.3-EDA-Livestocks.png}
\caption{2.1.3-EDA-Livestocks}
\end{figure}
\FloatBarrier
\begin{figure}[H]
\centering
\includegraphics[width=0.95\textwidth]{Code/2.1.4-Summary-Dataset.png}
\caption{2.1.4-Summary-Dataset}
\end{figure}
\FloatBarrier
\begin{figure}[H]
\centering
\includegraphics[width=0.95\textwidth]{Code/2.2.1-Data-Processing-District.png}
\caption{2.2.1-Data-Processing-District}
\end{figure}
\FloatBarrier
\begin{figure}[H]
\centering
\includegraphics[width=0.95\textwidth]{Code/2.2.2-Check-District-Mismatch.png}
\caption{2.2.2-Check-District-Mismatch}
\end{figure}
\FloatBarrier
\begin{figure}[H]
\centering
\includegraphics[width=0.95\textwidth]{Code/2.3-Merge-Datasets.png}
\caption{2.3-Merge-Datasets}
\end{figure}
\FloatBarrier
\begin{figure}[H]
\centering
\includegraphics[width=0.95\textwidth]{Code/2.3.1-Check-Merge-Missing-Values.png}
\caption{2.3.1-Check-Merge-Missing-Values}
\end{figure}
\FloatBarrier
\begin{figure}[H]
\centering
\includegraphics[width=0.95\textwidth]{Code/2.4-Feature-Engineering.png}
\caption{2.4-Feature-Engineering}
\end{figure}
\FloatBarrier
\begin{figure}[H]
\centering
\includegraphics[width=0.95\textwidth]{Code/2.4.2-Distribution.png}
\caption{2.4.2-Distribution}
\end{figure}
\FloatBarrier
\begin{figure}[H]
\centering
\includegraphics[width=0.95\textwidth]{Code/2.4.3-Pairplot.png}
\caption{2.4.3-Pairplot}
\end{figure}
\FloatBarrier
\begin{figure}[H]
\centering
\includegraphics[width=0.95\textwidth]{Code/3.0-Generate-Synthetic-Data.png}
\caption{3.0-Generate-Synthetic-Data}
\end{figure}
\FloatBarrier
\begin{figure}[H]
\centering
\includegraphics[width=0.95\textwidth]{Code/3.0.1-EDA-50-Year-Dataset.png}
\caption{3.0.1-EDA-50-Year-Dataset}
\end{figure}
\FloatBarrier
\begin{figure}[H]
\centering
\includegraphics[width=0.95\textwidth]{Code/3.0.2-Correlation-Analysis.png}
\caption{3.0.2-Correlation-Analysis}
\end{figure}
\FloatBarrier
\begin{figure}[H]
\centering
\includegraphics[width=0.95\textwidth]{Code/3.0.3-Distribution-Analysis.png}
\caption{3.0.3-Distribution-Analysis}
\end{figure}
\FloatBarrier
\begin{figure}[H]
\centering
\includegraphics[width=0.95\textwidth]{Code/3.0.4-Box-Plot.png}
\caption{3.0.4-Box-Plot}
\end{figure}
\FloatBarrier
\begin{figure}[H]
\centering
\includegraphics[width=0.95\textwidth]{Code/3.0.5-Statistical-Summary.png}
\caption{3.0.5-Statistical-Summary}
\end{figure}
\FloatBarrier
\begin{figure}[H]
\centering
\includegraphics[width=0.95\textwidth]{Code/3.2-Feature-Engineering.png}
\caption{3.2-Feature-Engineering}
\end{figure}
\FloatBarrier
\begin{figure}[H]
\centering
\includegraphics[width=0.95\textwidth]{Code/3.2.1-Time-Series-Visualization.png}
\caption{3.2.1-Time-Series-Visualization}
\end{figure}
\FloatBarrier
\begin{figure}[H]
\centering
\includegraphics[width=0.95\textwidth]{Code/3.2.2-Feature-Variance-Analysis.png}
\caption{3.2.2-Feature-Variance-Analysis}
\end{figure}
\FloatBarrier
\begin{figure}[H]
\centering
\includegraphics[width=0.95\textwidth]{Code/4.0-National-Trend-Analysis.png}
\caption{4.0-National-Trend-Analysis}
\end{figure}
\FloatBarrier
\begin{figure}[H]
\centering
\includegraphics[width=0.95\textwidth]{Code/4.0.1-50Years-Analysis.png}
\caption{4.0.1-50Years-Analysis}
\end{figure}
\FloatBarrier
\begin{figure}[H]
\centering
\includegraphics[width=0.95\textwidth]{Code/5.0-Era-Based-Clustering.png}
\caption{5.0-Era-Based-Clustering}
\end{figure}
\FloatBarrier
\begin{figure}[H]
\centering
\includegraphics[width=0.95\textwidth]{Code/5.1-Elbow-Method.png}
\caption{5.1-Elbow-Method}
\end{figure}
\FloatBarrier
\begin{figure}[H]
\centering
\includegraphics[width=0.95\textwidth]{Code/5.1.1-Elbow-Plot.png}
\caption{5.1.1-Elbow-Plot}
\end{figure}
\FloatBarrier
\begin{figure}[H]
\centering
\includegraphics[width=0.95\textwidth]{Code/5.2-Fit-K-Means.png}
\caption{5.2-Fit-K-Means}
\end{figure}
\FloatBarrier
\begin{figure}[H]
\centering
\includegraphics[width=0.95\textwidth]{Code/5.2.1-Characterize-Clusters.png}
\caption{5.2.1-Characterize-Clusters}
\end{figure}
\FloatBarrier
\begin{figure}[H]
\centering
\includegraphics[width=0.95\textwidth]{Code/5.2.2-Clusters-Visualization.png}
\caption{5.2.2-Clusters-Visualization}
\end{figure}
\FloatBarrier
\begin{figure}[H]
\centering
\includegraphics[width=0.95\textwidth]{Code/5.2.3-Clusters-HeatMap.png}
\caption{5.2.3-Clusters-HeatMap}
\end{figure}
\FloatBarrier
\begin{figure}[H]
\centering
\includegraphics[width=0.95\textwidth]{Code/5.2.4-Cluster-Evolution.png}
\caption{5.2.4-Cluster-Evolution}
\end{figure}
\FloatBarrier
\begin{figure}[H]
\centering
\includegraphics[width=0.95\textwidth]{Code/5.2.5-Clusters-Transition-Analysis.png}
\caption{5.2.5-Clusters-Transition-Analysis}
\end{figure}
\FloatBarrier
\begin{figure}[H]
\centering
\includegraphics[width=0.95\textwidth]{Code/6.0-Structural-Change.png}
\caption{6.0-Structural-Change}
\end{figure}
\FloatBarrier
\begin{figure}[H]
\centering
\includegraphics[width=0.95\textwidth]{Code/6.1-Visualize.png}
\caption{6.1-Visualize}
\end{figure}
\FloatBarrier
\begin{figure}[H]
\centering
\includegraphics[width=0.95\textwidth]{Code/7.0-Forecasting.png}
\caption{7.0-Forecasting}
\end{figure}
\FloatBarrier
\begin{figure}[H]
\centering
\includegraphics[width=0.95\textwidth]{Code/7.1-Visualize.png}
\caption{7.1-Visualize}
\end{figure}
\FloatBarrier
\begin{figure}[H]
\centering
\includegraphics[width=0.95\textwidth]{Code/8.0-Decision-Tree-Classifier.png}
\caption{8.0-Decision-Tree-Classifier}
\end{figure}
\FloatBarrier
\begin{figure}[H]
\centering
\includegraphics[width=0.95\textwidth]{Code/8.1-Model-Evaluation.png}
\caption{8.1-Model-Evaluation}
\end{figure}
\FloatBarrier
\begin{figure}[H]
\centering
\includegraphics[width=0.95\textwidth]{Code/8.2-Confusion-Matrix.png}
\caption{8.2-Confusion-Matrix}
\end{figure}
\FloatBarrier
\begin{figure}[H]
\centering
\includegraphics[width=0.95\textwidth]{Code/8.3-Visualization.png}
\caption{8.3-Visualization}
\end{figure}
\FloatBarrier
\begin{figure}[H]
\centering
\includegraphics[width=0.95\textwidth]{Code/8.4-Decision-Tree-Text.png}
\caption{8.4-Decision-Tree-Text}
\end{figure}
\FloatBarrier
\begin{figure}[H]
\centering
\includegraphics[width=0.95\textwidth]{Code/8.5-Feature-Importance.png}
\caption{8.5-Feature-Importance}
\end{figure}
\FloatBarrier
\begin{figure}[H]
\centering
\includegraphics[width=0.95\textwidth]{Code/8.6-Model-Performance-By-Era.png}
\caption{8.6-Model-Performance-By-Era}
\end{figure}
\FloatBarrier
\begin{figure}[H]
\centering
\includegraphics[width=0.95\textwidth]{Code/9.0-Policy-Recommendations.png}
\caption{9.0-Policy-Recommendations}
\end{figure}
\FloatBarrier
\begin{figure}[H]
\centering
\includegraphics[width=0.95\textwidth]{Code/10.0-Thematic-Map-Nepal.png}
\caption{10.0-Thematic-Map-Nepal}
\end{figure}
\FloatBarrier
\begin{figure}[H]
\centering
\includegraphics[width=0.95\textwidth]{Code/11.0-Summary.png}
\caption{11.0-Summary}
\end{figure}
\FloatBarrier

\end{document}
